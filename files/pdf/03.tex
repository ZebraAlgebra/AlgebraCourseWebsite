\documentclass{article}

\usepackage[english]{babel}

\usepackage[letterpaper,top=2cm,bottom=2cm,left=3cm,right=3cm,marginparwidth=1.75cm]{geometry}

\usepackage{amsmath}
\usepackage{amssymb}
\usepackage{tikz-cd}
\usepackage{graphicx}
\usepackage[colorlinks=true, allcolors=blue]{hyperref}

\begin{document}

\section*{Problem 1}\paragraph{}
Let $A$ be a local ring, $M$ and $N$ finitely generated $A$-modules.

Prove that if $M \otimes N=0$, then $M=0$ or $N=0$.

[Let $\mathfrak{m}$ be the maximal ideal, $k=A / \mathfrak{m}$ the residue field.

Let $M_k=k \otimes_A M \cong$ $M / \mathfrak{m} M$ by Exercise 2.

By Nakayama's lemma, $M_k=0 \Rightarrow M=0$.

But $M \otimes_A N=0 \Rightarrow\left(M \otimes_A N\right)_k=0 \Rightarrow M_k \otimes_k N_k=0 \Rightarrow M_k=0$ or $N_k=0$,
since $M_k, N_k$ are vector spaces over a field.]

\section*{Problem 2}\paragraph{}
Let $M$ be a finitely generated $A$-module and $\phi: M \rightarrow A^n$ a surjective homomorphism. Show that $\operatorname{Ker}(\phi)$ is finitely generated.

[Let $e_1, \ldots, e_n$ be a basis of $A^n$ and choose $u_i \in M$ such that $\phi\left(u_i\right)=e_i$ $(1 \leqslant i \leqslant n)$. Show that $M$ is the direct sum of $\operatorname{Ker}(\phi)$ and the submodule generated by $\left.u_1, \ldots, u_n \cdot\right]$

\section*{Problem 3}\paragraph{}
Let $\mathfrak{a}$ be an ideal of a ring $A$, and let $S=1+\mathfrak{a}$.

Show that $S^{-1} \mathfrak{a}$ is contained in the Jacobson radical of $S^{-1} A$.

Use this result and Nakayama's lemma to give a proof of (2.5) which does not depend on determinants.

[If $M=\mathfrak{a} M$, then $S^{-1} M=\left(S^{-1} \mathfrak{a}\right)\left(S^{-1} M\right)$, hence by Nakayama we have $S^{-1} M=0$. Now use Exercise 1.]

\section*{Problem 4}\paragraph{}
The set $S_0$ of all non-zero-divisors in $A$ is a saturated multiplicatively closed subset of $A$.

Hence the set $D$ of zero-divisors in $A$ is a union of prime ideals (see Chapter 1, Exercise 14).

Show that every minimal prime ideal of $A$ is contained in D. [Use Exercise 6.]

The ring $S_0^{-1} A$ is called the \textit{total ring of fractions} of $A$.

Prove that

\begin{itemize}
    \item i) $S_0$ is the largest multiplicatively closed subset of $A$ for which the homomorphism $A \rightarrow S_0^{-1} A$ is injective.

    \item ii) Every element in $S_0^{-1} A$ is either a zero-divisor or a unit.

    \item iii) Every ring in which every non-unit is a zero-divisor is equal to its total ring of fractions (i.e., $A \rightarrow S_0^{-1} A$ is bijective).
\end{itemize}


\section*{Problem 5}\paragraph{}
Let $\varphi: A \rightarrow B$ be a commutative ring homomorphism.

Let $E$ be an $A$-module and $F$ a $B$-module.

Let $F_A$ be the $A$-module obtained from $F$ via the operation of $A$ on $F$ through $\varphi$, that is for $y \in F_A$ and $a \in A$ this operation is given by

$$
(a, y) \mapsto \varphi(a) y .
$$

Show that there is a natural isomorphism

$$
\operatorname{Hom}_B\left(B \otimes_A E, F\right) \approx \operatorname{Hom}_A\left(E, F_A\right)
$$

\section*{Problem 6}\paragraph{}
[The norm]

Let $B$ be a commutative algebra over the commutative ring $R$ and assume that $B$ is free of rank $r$.

Let $A$ be any commutative $R$-algebra.

Then $A \otimes B$ is both an $A$-algebra and a $B$-algebra.

We view $A \otimes B$ as an $A$-algebra, which is also free of rank $r$.

If $\left\{e_1, \ldots, e_r\right\}$ is a basis of $B$ over $R$, then

$$
1_A \otimes e_1, \ldots, 1_A \otimes e_r
$$

is a basis of $A \otimes B$ over $A$.

We may then define the norm

$$
N=N_{A \otimes B, A}: A \otimes B \rightarrow A
$$

as the unique map which coincides with the determinant of the regular representation.

In other words, if $b \in B$ and $b_B$ denotes multiplication by $b$, then

$$
N_{B, R}(b)=\operatorname{det}\left(b_B\right)
$$

and similarly after extension of the base. Prove:

\begin{itemize}
    \item (a) Let $\varphi: A \rightarrow C$ be a homomorphism of $R$-algebras. Then the following diagram is commutative:
\[\begin{tikzcd}
	{A\otimes{B}} & {C\otimes{B}} \\
	A & C
	\arrow["\varphi\otimes1", from=1-1, to=1-2]
	\arrow["N", from=1-2, to=2-2]
	\arrow["N"', from=1-1, to=2-1]
	\arrow["\varphi"', from=2-1, to=2-2]
\end{tikzcd}\]
    \item (b) Let $x, y \in A \otimes B$. Then $N\left(x \otimes_B y\right)=N(x) \otimes N(y)$. [Hint: Use the commutativity relations $e_i e_j=e_j e_i$ and the associativity.]
\end{itemize}


\end{document}
