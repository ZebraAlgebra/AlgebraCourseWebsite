% Problem 1
Let $M$ be a module over the integral domain $R$.
\begin{itemize}
    \item (a) Suppose that $M$ has rank $n$ and that $x_1, x_2, \ldots, x_n$ is any maximal set of linearly independent elements of $M$.

Let $N=R x_1+\ldots+R x_n$ be the submodule generated by $x_1, x_2, \ldots, x_n$.

Prove that $N$ is isomorphic to $R^n$ and that the quotient $M / N$ is a torsion $R$-module (equivalently, the elements $x_1, \ldots, x_n$ are linearly independent and for any $y \in M$ there is a nonzero element $r \in R$ such that $r y$ can be written as a linear combination $r_1 x_1+\ldots+r_n x_n$ of the $\left.x_i\right)$.

    \item (b) Prove conversely that if $M$ contains a submodule $N$ that is free of rank $n$ (i.e., $N \cong$ $R^n$ ) such that the quotient $M / N$ is a torsion $R$-module then $M$ has rank $n$.

[Let $y_1, y_2, \ldots, y_{n+1}$ be any $n+1$ elements of $M$.

Use the fact that $M / N$ is torsion to write $r_i y_i$ as a linear combination of a basis for $N$ for some nonzero elements $r_1, \ldots, r_{n+1}$ of $R$.

Use an argument as in the proof of Proposition 3 to see that the $r_i y_i$, and hence also the $y_i$, are linearly dependent.]

\end{itemize}

% Problem 2
Let $R$ be any ring, let $A_1, A_2, \ldots, A_m$ be $R$-modules and let $B_i$ be a submodule of $A_i$, $1 \leq i \leq m$.

Prove that

$$
\left(A_1 \oplus A_2 \oplus \cdots \oplus A_m\right) /\left(B_1 \oplus B_2 \oplus \cdots \oplus B_m\right) \cong\left(A_1 / B_1\right) \oplus\left(A_2 / B_2\right) \oplus \cdots \oplus\left(A_m / B_m\right) \text {.
}
$$

% Problem 3
Let $R$ be a P.I.D., let $a$ be a nonzero element of $R$ and let $M=R /(a)$.

For any prime $p$ of $R$ prove that

$$
p^{k-1} M / p^k M \cong \begin{cases}R /(p) & \text { if } k \leq n \\ 0 & \text { if } k>n\end{cases}
$$

where $n$ is the power of $p$ dividing $a$ in $R$.

% Problem 4
The following set of exercises outlines a proof of Theorem 5 in the special case where $R$ is a Euclidean Domain using a matrix argument involving row and column operations.

This applies in particular to the cases $R=\mathbb{Z}$ and $R=F[x]$ of interest in the applications and is computationally useful.

Let $R$ be a Euclidean Domain and let $M$ be an $R$-module.

Prove that $M$ is finitely generated if and only if there is a surjective $R$-homomorphism $\varphi: R^n \rightarrow M$ for some integer $n$ (this is true for any ring $R$ ).

% Problem 5
Determine all possible rational canonical forms for a linear transformation with characteristic polynomial $x^2\left(x^2+1\right)^2$.