% Problem 1

Let $A$ be a local ring, $M$ and $N$ finitely generated $A$-modules.

Prove that if $M \otimes N=0$, then $M=0$ or $N=0$.

[Let $\mathfrak{m}$ be the maximal ideal, $k=A / \mathfrak{m}$ the residue field.

Let $M_k=k \otimes_A M \cong$ $M / \mathfrak{m} M$ by Exercise 2.

By Nakayama's lemma, $M_k=0 \Rightarrow M=0$.

But $M \otimes_A N=0 \Rightarrow\left(M \otimes_A N\right)_k=0 \Rightarrow M_k \otimes_k N_k=0 \Rightarrow M_k=0$ or $N_k=0$,
since $M_k, N_k$ are vector spaces over a field.]

% Problem 2

Let $M$ be a finitely generated $A$-module and $\phi: M \rightarrow A^n$ a surjective homomorphism. Show that $\operatorname{Ker}(\phi)$ is finitely generated.

[Let $e_1, \ldots, e_n$ be a basis of $A^n$ and choose $u_i \in M$ such that $\phi\left(u_i\right)=e_i$ $(1 \leqslant i \leqslant n)$. Show that $M$ is the direct sum of $\operatorname{Ker}(\phi)$ and the submodule generated by $\left.u_1, \ldots, u_n \cdot\right]$

% Problem 3

Let $\mathfrak{a}$ be an ideal of a ring $A$, and let $S=1+\mathfrak{a}$.

Show that $S^{-1} \mathfrak{a}$ is contained in the Jacobson radical of $S^{-1} A$.

Use this result and Nakayama's lemma to give a proof of (2.5) which does not depend on determinants.

[If $M=\mathfrak{a} M$, then $S^{-1} M=\left(S^{-1} \mathfrak{a}\right)\left(S^{-1} M\right)$, hence by Nakayama we have $S^{-1} M=0$. Now use Exercise 1.]

% Problem 4

The set $S_0$ of all non-zero-divisors in $A$ is a saturated multiplicatively closed subset of $A$.

Hence the set $D$ of zero-divisors in $A$ is a union of prime ideals (see Chapter 1, Exercise 14).

Show that every minimal prime ideal of $A$ is contained in D. [Use Exercise 6.]

The ring $S_0^{-1} A$ is called the \textit{total ring of fractions} of $A$.

Prove that

\begin{itemize}
    \item i) $S_0$ is the largest multiplicatively closed subset of $A$ for which the homomorphism $A \rightarrow S_0^{-1} A$ is injective.

    \item ii) Every element in $S_0^{-1} A$ is either a zero-divisor or a unit.

    \item iii) Every ring in which every non-unit is a zero-divisor is equal to its total ring of fractions (i.e., $A \rightarrow S_0^{-1} A$ is bijective).
\end{itemize}

% Problem 5

Let $A$ be a principal entire ring, and let $K$ be its quotient field.

Let $\mathfrak{o}$ be a valuation ring of $K$ containing $A$, and assume $\mathfrak{o} \neq K$.

Show that $\mathfrak{o}$ is the local ring $A_{(p)}$ for some prime element $p$.

[This applies both to the ring $\mathbb{Z}$ and to a polynomial ring $k[X]$ over a field $k$.]

% Problem 6

Let $A$ be an entire ring, and let $K$ be its quotient field.

Assume that every finitely generated ideal of $A$ is principal.

Let $\mathfrak{o}$ be a discrete valuation ring of $K$ containing $A$.

Show that $\mathfrak{o}=A_{(p)}$ for some element $p$ of $A$, and that $p$ is a generator of the maximal ideal of $\mathfrak{o}$.
