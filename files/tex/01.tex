% Problem 1

A partially ordered set $I$ is set to be a \textit{directed set} if for each pair $i, j$ in $I$ there exists $k\in I$ such that $i\leq k$ and $j\leq k$.

Let $A$ be a ring, let $I$ be a direct set and let $(M_i)_{i\in I}$ be a family of $A$-modules indexed by $I$. For each pair $i, j$ in $I$ such that $i\leq j$, let $\mu_{ij}:M_i\to M_j$ be an $A$-homomorphism, and suppose that the following axioms are satisfied:

\begin{enumerate}
    \item $\mu_{ii}$ is the identity mapping of $M_i$ for all $i\in I$.
    \item $\mu_{ik}=\mu_{jk}\circ\mu_{ij}$ whenever $i\leq j\leq k$.
\end{enumerate}

Then the modules $M_i$ and homomorphisms $\mu_{ij}$ are said to form a \textit{direct system} $\mathbf{M}=(M_i,\mu_{ij})$ over the directed set $I$.

We shall construct an $A$-module $M$ called the \textit{direct limit} of the direct system $\mathbf{M}$. Let $C$ be the direct sum of the $M_i$, and identify each module $M_i$, with its canonical image in $C$. Let $D$ be the submodule of $C$ generated by all elements of the form $x_i-\mu_{ij}(x_i)$ where $i\leq j$ and $x_i\in M_i$. Let $M = C/D$, let $\mu:C\to M$ be the projection and let $\mu_i$ be the restriction of $\mu$ to $M_i$.

The module $M$, or more correctly the pair consisting of $M$ and the family of homomorphisms $\mu_i:M_i\to M$ is called the \textit{direct limit} of the direct system $\mathbf{M}$, and is written $\varinjlim M_i$. From the construction it is clear that $\mu_i=\mu_j\circ\mu_{ij}$ whenever $i\leq j$.

% Problem 2

In the situation of Exercise 14, show that every element of $M$ can be written in the form $\mu_i(x_i)$ for some $i\in I$ and some $x_i\in M_i$.

Show also that if $\mu_i(x_i)=0$ then there exists $j\geq i$ such that $\mu_{ij}(x_i)=0$ in $M_j$.

% Problem 3

Show that the direct limit is characterized (up to isomorphism) by the following property. Let $N$ be an $A$-module and for each $i\in I$ let $\alpha_i :M_i\to N$ be an $A$-module homomorphism such that $\alpha_i = \alpha_j\circ\mu_{ij}$ whenever $i \leq j$. Then there exists a unique homomorphism $\alpha: M\to N$ such that $\alpha_i=\alpha\circ\mu_i$ for all $i\in I$.
